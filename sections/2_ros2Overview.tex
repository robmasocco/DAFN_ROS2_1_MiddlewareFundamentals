% Section 2 - ROS 2 Overview
% Roberto Masocco <roberto.masocco@uniroma2.it>
% March 13, 2022

% ### ROS 2 Overview ###
\section{ROS 2 Overview}
\graphicspath{{figs/section2/}}

% --- What is ROS 2? ---
\begin{frame}{What is ROS 2?}
\begin{columns}
    \column{.5\textwidth}
    \only<1>{
        \nolistindent ROS 2 is a \textbg{DDS-based}, \textbg{open-source} middleware for robotic applications. It allows developers to build and manage \textbg{distributed control architectures} made of many modules, usually referred to as \textbg{nodes}.
    }
    \only<2>{
        \nolistindent \nolistindent ROS 2 currently supports \textbg{C++} and \textbg{Python} for application programming, and runs natively on \textbg{Ubuntu Linux 20.04}.\\
        New versions are periodically released as \textbg{distributions}: the current LTS one is \textbg{Foxy Fitzroy} and the latest one today is \textbg{Galactic Geochelone}; the development version is \textbg{Rolling Ridley} and can only be compiled from source.
    }

    \column{.5\textwidth}
    \begin{figure}
        \centering
        \includegraphics[width=.9\textwidth]{ros2Logo.jpg}
        \caption{ROS 2 logo}
        \label{fig:ros2logos}
    \end{figure}
\end{columns}
\end{frame}

% --- Main Features ---
\begin{frame}{Main Features}
As a middleware, it offers the following services to roboticists:
\begin{itemize}
    \item<1-> \textbg{three inter-process communication (IPC) paradigms}, easy to set up and based on the DDS;
    \item organization of software packages, allowing for \textbg{redistribution and code reuse}, thanks to the \textbg{colcon} package manager;
    \item<1-> module configuration tools: \textbg{node parameters} and \textbg{launch files};
    \item<1-> integrated \textbg{logging subsystem} (involves both console and log files);
    \item<1-> CLI \textbg{introspection tools} for debugging and testing;
    \item<1-> integration with \textbg{simulators} (e.g. Gazebo) and \textbg{data visualizers} (e.g. RViz);
    \item<2> and much more...
\end{itemize}
\end{frame}

% --- Flaws ---
\begin{frame}{Flaws}
\visible<1->{
    \begin{alertblock}{ROS 2 biggest flaws (as of today)}
        The main concerns arise when developing low-level stuff:
        \begin{itemize}
            \item the DDS layer is almost completely abstracted, so some network configurations are impossible;
            \item the internal job scheduling algorithm (the \textbr{executor}) is \textbr{not suited for hard real-time applications}.
        \end{itemize}
    \end{alertblock}
}
\visible<2>{
    What to do when development gets to a really low level?
    \begin{itemize}
        \item Use \textbg{micro-ROS}: ROS 2 on microcontrollers and different communication interfaces.
        \item Hand off stuff to \textbg{microcontrollers}.
        \item Use something else.
    \end{itemize}
}
\end{frame}

% --- Job Executor ---
\begin{frame}{Job Executor}
TODO:
\begin{itemize}
    \item event-based explanation
    \item node jobs+executor+callback scheme
    \item "to be addressed in new upcoming releases+micro-ROS"
\end{itemize}
\end{frame}
